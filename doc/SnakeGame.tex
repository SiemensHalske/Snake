\documentclass{article}
\usepackage[utf8]{inputenc}
\usepackage{geometry}
\usepackage{enumitem}

\geometry{a4paper, margin=1in}

\title{Entwicklungsplan für das Snake-Spiel}
\author{Jan Osing, Gabriel Janich, \\ Hendrik Siemens}
\date{\today}

\begin{document}
	
	\maketitle
	
	\section{Projekt Setup und Vorbereitung}
	\textbf{Ziele:}
	\begin{itemize}
		\item Einrichten der Entwicklungsumgebung.
		\item Festlegung der Projektstruktur und Dateiorganisation.
	\end{itemize}
	\textbf{Aufgaben:}
	\begin{enumerate}
		\item Webserver (z.B. Apache oder Nginx).
		\item MySQL-Datenbankserver einrichten.
		\item PHP-Entwicklungsumgebung konfigurieren (z.B. XAMPP, LAMP).
		\item Verzeichnisstruktur für das Projekt erstellen.
	\end{enumerate}
	
	\section{Datenbankdesign und -integration}
	\textbf{Ziele:}
	\begin{itemize}
		\item Erstellen und Konfigurieren der MySQL-Datenbank.
		\item Sicherstellen, dass die Datenbankverbindung effizient und sicher ist.
	\end{itemize}
	\textbf{Aufgaben:}
	\begin{enumerate}
		\item Ausführen der SQL-Skripte zur Erstellung der Datenbank und Tabellen.
		\item PHP-Skripte schreiben, um eine Datenbankverbindung herzustellen.
		\item CRUD-Operationen für Spieler und Scores als PHP-Funktionen implementieren.
	\end{enumerate}
	
	\section{Backend-Entwicklung}
	\textbf{Ziele:}
	\begin{itemize}
		\item Erstellen der Server-Logik zur Verwaltung von Spielsessions und Spielerdaten.
	\end{itemize}
	\textbf{Aufgaben:}
	\begin{enumerate}
		\item PHP-Skripte entwickeln für:
		\begin{itemize}
			\item Anmeldung und Registrierung von Spielern.
			\item Starten und Beenden von Spielsitzungen.
			\item Aktualisierung und Abruf von Spieler-Scores.
		\end{itemize}
	\end{enumerate}
	
	\section{Frontend-Entwicklung}
	\textbf{Ziele:}
	\begin{itemize}
		\item Design und Implementierung der Benutzeroberfläche.
		\item Entwicklung der Spiellogik.
	\end{itemize}
	\textbf{Aufgaben:}
	\begin{enumerate}
		\item HTML-Seiten erstellen für Spielansicht, Leaderboard und Spielerprofil.
		\item CSS für responsive Design verwenden.
		\item JavaScript für die Spiellogik entwickeln.
		\item AJAX verwenden, um die Highscore-Tabelle live zu aktualisieren.
	\end{enumerate}
	
	\section{Testing und Qualitätssicherung}
	\textbf{Ziele:}
	\begin{itemize}
		\item Sicherstellen, dass das Spiel auf verschiedenen Geräten und Browsern korrekt funktioniert.
		\item Überprüfung der Sicherheit, insbesondere der Datenübertragungen.
	\end{itemize}
	\textbf{Aufgaben:}
	\begin{enumerate}
		\item Funktionstests durchführen für alle Features.
		\item Performance-Tests für die Webseite und die Datenbank.
		\item Sicherheitstests, um SQL-Injection und XSS-Angriffe zu verhindern.
	\end{enumerate}
	
	\section{Deployment und Wartung}
	\textbf{Ziele:}
	\begin{itemize}
		\item Veröffentlichung der Anwendung auf einem Live-Server.
		\item Planung für die Wartung und mögliche Updates.
	\end{itemize}
	\textbf{Aufgaben:}
	\begin{enumerate}
		\item Auswählen eines Hosting-Anbieters und Konfigurieren des Live-Servers.
		\item Deployment der Anwendung und der Datenbank auf dem Live-Server.
		\item Einrichten eines VCS (z.B. Git) für Versionskontrolle.
		\item Überwachung der Anwendung auf Fehler und Performance-Probleme.
	\end{enumerate}

\section{Authentifizierung und Sicherheit}
\textbf{Ziele:}
\begin{itemize}
	\item Implementierung einer sicheren Benutzerauthentifizierung und -autorisation.
	\item Schutz der Benutzerdaten und -interaktionen innerhalb der Anwendung.
\end{itemize}
\textbf{Aufgaben:}
\begin{enumerate}
	\item Implementierung des Registrierungsprozesses für neue Benutzer:
	\begin{itemize}
		\item Entwicklung eines sicheren Registrierungs-Endpoints.
		\item Verwendung von Passwort-Hashing mit modernen Sicherheitsstandards.
	\end{itemize}
	\item Einrichtung des Login-Prozesses unter Verwendung von JSON Web Tokens (JWT):
	\begin{itemize}
		\item Integration der Firebase/php-jwt-Bibliothek zur Token-Erzeugung und -Verifizierung.
		\item Erstellung von Token nach erfolgreicher Authentifizierung und Rückgabe an den Client.
		\item Validierung des Tokens bei jedem authentifizierungspflichtigen Request.
	\end{itemize}
	\item Implementierung des Logout-Mechanismus:
	\begin{itemize}
		\item Anleitung des Clients, das gespeicherte Token zu löschen.
		\item Optional: Einrichtung einer Token-Blacklist auf dem Server zur Invalidierung von Tokens.
	\end{itemize}
	\item Sicherstellung der Sicherheit durch HTTPS für alle Kommunikationswege.
	\item Durchführung von Sicherheitstests, um die Wirksamkeit der implementierten Maßnahmen zu überprüfen.
\end{enumerate}

\section{Changelog und Dokumentation der Entwicklung}
\textbf{Einträge:}
\begin{itemize}
    \item \textbf{2024-04-19:} 
        \begin{itemize}
            \item Projektinitialisierung und Setup der Entwicklungsumgebung.
            \item Einrichtung von Apache, MySQL, und PHP auf einem Raspberry Pi 4B.
            \item Beginn der Erstellung der Datenbank und der notwendigen Tabellen für das Spiel.
            \item Einrichtung und Konfiguration von SSL für sichere Datenübertragungen.
            \item Erste Schritte zur Konfiguration von Apache und PHP.
        \end{itemize}
    \item \textbf{2024-04-20:}
        \begin{itemize}
            \item Implementierung des Login-Endpunkts unter Verwendung von JWT für die Authentifizierung.
            \item Identifizierung und Lösung von Problemen mit der Apache-Konfiguration.
            \item Erste Tests der Authentifizierungsfunktionen durchgeführt.
        \end{itemize}
    \item \textbf{2024-04-20:} 
        \begin{itemize}
            \item Implementierung des Registrierungs-Endpunkts und Integration des sicheren Passwort-Hashing.
            \item Implementierung des Logout-Mechanismus, der das Löschen des Tokens auf der Client-Seite umfasst.
        \end{itemize}
\end{itemize}

\vspace{1em}
\textbf{Dokumentation:} \newline
Die Entwicklungsdokumentation umfasst technische Spezifikationen, Code-Kommentare und Setup-Anleitungen, um eine konsistente Wartung und zukünftige Erweiterungen zu gewährleisten. Eine detaillierte Dokumentation der Authentifizierungsprozesse wurde ebenfalls hinzugefügt, um die Sicherheitsmechanismen und den Datenschutz innerhalb der Anwendung zu erläutern.



	
\end{document}
